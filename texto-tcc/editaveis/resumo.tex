\begin{resumo}

As ferramentas de análise estática isoladas não proveem bons mecanismos de visualização de dados para transmitir a informação da qualidade do principal produto do desenvolvimento de software, o código-fonte. Após uma revisão bibliográfica da literatura, identificou-se que ambientes de \textit{data warehousing} (DWing) são boas soluções em consultas e visualização de dados.Partindo desta premissa, este trabalho construiu um ambiente de DWing para visualização e extração de métricas de código-fonte, que são indicadores objetivos da qualidade deste. Por meio da realização de um estudo de caso de avaliação das métricas de código-fonte de um software livre no ambiente de DWing, verificou-se que ambientes de \textit{DWing} proveem melhores mecanismos de visualização e consulta de dados.


 \vspace{\onelineskip}
    
 \noindent
 \textbf{Palavras-chaves}: Métricas de Código-Fonte. \textit{Data Warehousing}. \textit{Data Warehouse}
\end{resumo}
