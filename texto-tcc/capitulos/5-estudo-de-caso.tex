\chapter{Estudo de Caso}
\label{estudo de caso}


\citeonline{wohlin2012experimentation} enuncia que é necessário que uma metodologia de estudo de caso, definida formalmente em engenharia de software, deve conter os passos da Figura \ref{fig:metodologia-estudo}.

\begin{figure}[ht!]
\centering
\includegraphics[keepaspectratio=true,scale=0.16]{figuras/metodologia-estudo-caso.eps}
\caption{Metodologia de Estudo de Caso proposta por \citeonline{wohlin2012experimentation}}
\label{fig:metodologia-estudo}
\end{figure}
\FloatBarrier

\section{Planejamento do Estudo de Caso}

Para o planejamento do estudo de caso em engenharia de software é essencial, segundo \citeonline{brereton2008using},descrever o protocolo do estudo de caso. Este consiste em um relatório simplificado das principais variáveis, dados e passos, tais como o meio e modo de coleta de dados, indetificação das fontes de dados e formas de análise de dados \cite{wohlin2012experimentation}. 


O protocolo de estudo de caso deste trabalho, que foi baseado em \citeonline{brereton2008using} e \citeonline{wohlin2012experimentation} foi dividido em seções, conforme se vê a seguir.


\section{Trabalhos Relacionados ao Estudo de Caso}

Assim como os trabalhos de \citeonline{Palza2003}, \citeonline{Ruiz2005}, \citeonline{Castellanos2005}, \citeonline{Becker2006}, \citeonline{Folleco2007}, \citeonline{Silveira2010}, buscou-se alcançar uma validação empírica da utilização ambiente de \textit{Data Warehousing} proposto no Capítulo \ref{chap:arquitetura}.


\subsection{Objetivos do Estudo de Caso}

O objetivo geral do estudo de caso foi avaliar um software durante o seu desenvolvimento no ambiente de \textit{Data Warehousing}. Em consônância ao objetivo geral do estudo de caso e aos objetivos específicos deste trabalho, foram definidos objetivos específicos do estudo de caso, tal como se mostra na Tabela \ref{tab:objetivos-estudo-de-caso}.

\begin{table}[H]
\begin{center}
\input{tabelas/objetivos-estudo.ltx}
\caption{Objetivos Específicos de Estudo de Caso}
\label{tab:objetivos-estudo-de-caso}
\end{center}
\end{table}
\FloatBarrier

\subsection{Critérios de Seleção de Objeto do Estudo de Caso} 

Para selecionar o software que seria parte do objeto do estudo de caso considerou-se os 
critérios estábelecidos na Tabela \ref{tab:critérios-estudo-de-caso}. 


\begin{table}[H]
\begin{center}
\input{tabelas/criterios-estudo.ltx}
\caption{Critérios de Seleção de Objeto do Estudo de Caso}
\label{tab:critérios-estudo-de-caso}
\end{center}
\end{table}
\FloatBarrier

\subsection{Dados, Fonte dos Dados e Forma de Análise dos Dados}

Os principais dados quantitativos e qualitativos foram identificados na Tabela \ref{tab:dados-estudos}.

\begin{table}[H]
\begin{center}
\input{tabelas/dados-estudo.ltx}
\caption{Dados do Estudo de Caso}
\label{tab:dados-estudos}
\end{center}
\end{table}
\FloatBarrier



\section{Execução do Estudo de Caso e Análise dos Dados}