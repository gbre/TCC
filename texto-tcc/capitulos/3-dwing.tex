% -- Link com de Métricas (Registro Histórico, Comparações, Agregações, Tomada de Decisão)
\chapter{DWing}

Os principais fatores para a adoção de um programa de métricas em organizações 
de desenvolvimento de software são:
\begin{inparaenum}[i)]
\item a regularidade da coleta de dados;
\item a utilização de uma metodologia eficiente e transparente nessa coleta; 
\item o uso de ferramentas (não-intrusivas) para automatizar a coleta;
\item o uso de mecanismos de comunicação de resultados adequados para todos os 
envolvidos;
\item o uso de sofisticadas técnicas de análise de dados. 
\end{inparaenum}
\cite{Gopal2005} \cite{Silveira2010}.

\todo[inline, color=green!40]{Descrever Cada um dos itens e mudar o texto}


A definição mais difundida entre os pesquisadores apresenta o \textit{Data 
Warehouse (DW)} com um conjunto de dados integrados, consolidados, históricos, 
segmentados por assunto, não-voláteis, variáveis em relação ao tempo, e de 
apoio às decisões gerenciais \cite{Inmon1992}. Em uma outra definição de 
\textit{Data Warehouse} (DW) é um repositório central e consolidado 
\cite{Kimball2002}.


	
\section{Arquitetura de um \textit{Data Warehouse}} 
\todo[inline, color=green!40]{Colocar Figura}
\subsection{\textit{Extraction-Transformation-Load}}

As etapas de extração, transformação, carga e atualização do DW, formam o seu
back-end e caracterizam o processo chamado Extraction-Transform-Load (ETL). 
Esse processo pode ser dividido em três etapas distintas que somadas podem 
podendo consumir até 85\% de todo o esforço em um DWing \cite{Kimball2002}.

\begin{easylist}[itemize]

& Extração: No ambiente de \textit{data warehouse}, os dados são inicialmente 
extraídos de fontes externas de dados para um ambiente de \textit{staging} que 
\citeonline{Kimball2002} considera com uma área de armazenamento intermediária 
entre fontes e o \textit{data warehouse}. Normalmente, é de natureza 
temporária e o seu conteúdo é apagado após a carga dos dados no DW. 


Os dados podem provir de fontes distintas tais como planilhas, bases relacionais
em diferentes tipos de banco de dados (MySQL, Oracle, Postgres e etc) ou mesmo 
de webservices. 


& Transformação :

& Carga:

\end{easylist}
 

\subsection{\textit{On-Line Analytical Processing} (OLAP)}

O termo OLAP, inicialmente proposto por \citeonline{Codd1993}, 
é utilizado para caracterizar as aplicações voltadas ao suporte de atividades 
de análise, com o objetivo de prover a visualização dos dados sob diferentes 
perspectivas gerenciais e comportar todas as atividades de análise.

As ferramentas de apoio às consultas On-Line Analytical Processing (OLAP) ao DW 
representam o seu front-end.

\todo[inline, color=green!40]{Citar Consultas OLAP} 

\subsection{Armazém de Dados}
\todo[inline, color=green!40]{Falar Sobre Data Mart}
\todo[inline, color=green!40]{Falar Sobre Modelagem Mutidimensional}

\subsection{\textit{Dashboard}}




\section{Projeto de DW} 

\citeonline{Kimball2002} recomenda uma metodologia para o desenvolvimento
do projeto de um \textit{data warehouse}. 
Trata-se de uma abordagem de baixo para cima, que no caso de um 
\textit{data warehouse} que constrói um data mart por vez. As quatro etapas de 
projeto são:

\newlist{steps}{enumerate}{4}
\setlist[steps]{ label* = \arabic* -}

\begin{steps}
	\item Selecionar o Processo de Negócio com requisito fundamental do projeto 
	DW;
	
	\item Declarar a granularidade dos dados necessários para processo de 
	negócio, isto é, verificar a periodicidade dos dados 
	(diário, semanal, mensal, anual);
	
	\item Escolher as dimensões;
	
	\item Identificar os fatos;
	
	\item Apresentar, em formas visuais, as análises obtidas das métricas de 
	código-fonte;
	
    \end{steps}


	Seguindo esta metodologia proposta anteriormente

	