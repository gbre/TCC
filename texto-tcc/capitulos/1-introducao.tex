\chapter{Introdução}

As metodologias ágeis vem ganhando cada vez mais espaço no mercado global de desenvolvimento de software, pois enfatizam a qualidade do produto sobre a qualidade do processo, procurando minimizar a execução de atividades não essenciais ao longo do ciclo de vida de desenvolvimento de software.

O eXtreme Programming (XP), que é uma metodologia ágil, tem a codificação como atividade chave durante um projeto de desenvolvimento de software \cite{beck1999}. Isso se torna perceptível quando se analisa algumas práticas do XP,~\footnote{Documentação disponível em \url{http://www.extremeprogramming.org}} tais como:

\newlist{pratices}{enumerate}{4}
\setlist[pratices]{ label* = \arabic* -}


\begin{pratices}

\item Padronização do Código: O código é a principal forma de comunicação entre 
a equipe. A padronização de código o torna consistente e fácil para leitura e 
refatoração por todo o time.

\item Propriedade Coletiva do Código: Cada programador pode melhorar qualquer 
parte do código quando houver a oportunidade.

\item Programação em Pares: Todo código é escrito por duas pessoas: uma que 
olha para uma máquina, e outra com um teclado e um mouse.

\item Integração Contínua: Todo novo código é integrado ao sistema e, quando 
integrado, o sistema é totalmente reconstruído do zero e todos os testes devem 
passar. Caso contrário, o código é descartado.

\end{pratices} 

Dada a importância do código-fonte, infere-se a qualidade do trabalho produzido
 pela qualidade do código-fonte, sendo um dos métodos mais utilizados
para tal a análise estática de código-fonte. Durante anos, vários trabalhos 
foram publicados visando a definição formal de métodos de análise estática 
de código-fonte. Vide os trabalhos de \citeonline{Wichmann95}, 
\citeonline{Nielson:1999}, \citeonline{Emanuelsson2008}.
Contudo as ferramentas que realizam o procedimento no código-fonte ainda 
apresentam problemas, tais como:

\newlist{problems}{enumerate}{4}
\setlist[problems]{ label* = P\arabic* -}

\begin{problems}
    \item Ausência de resultados consolidados do produto, pois a maior 
	parte das métricas é extraída de elementos internos menores (Bibliotecas, 
	Pacotes, Classes, Métodos) do código-fonte.
    
	\item Ausência de mecanismos de tratamento, separação, recuperação, 
	organização e persistência de dados. 
	
	\item Ausência de associação entre resultados numéricos e forma de 
	interpretá-los: Ferramentas de análise estática frequentemente mostram 
	seus resultados como valores numéricos isolados para cada métrica 
	\cite{Meirelles2013}. 
	
	\item Em grande parte das ferramentas, a visualização dos resultados não é 
	agradável, isto é, são apresentados um conjunto de dados em uma janela 
	terminal contendo os valores das métricas.
	
    \end{problems}
	
Os problemas enunciados acima trazem muitos prejuízos às organizações que 
utilizam processos de aferição de qualidade de código-fonte como um indicador do desenvolvimento de produtos de software, pois sem possibilidade de manter 
registros das métricas de código-fonte, torna-se inviável qualquer análise
temporal da evolução da qualidade do código-fonte. Além disso, se os dados não 
são representativos, qualquer análise fica a cargo de experiências anteriores 
com as métricas de código-fonte.

Dado este contexto, é crucial que dados relacionados às métricas sejam 
coletados e compartilhados entre projetos e pessoas em uma visão organizacional unificada, para que determinada organização ou time possa compreender o processo de medição e monitoramento de projetos de software e, 
consequentemente, se tornar mais hábil e eficiente em realizar atividades 
técnicas relacionadas ao processo de desenvolvimento de software 
\cite{Chulani2003}. 



Vários trabalhos têm mostrado que ambientes de \textit{Data Warehousing} 
(DWing) são boas soluções para atender à visão organizacional unificada de 
métricas de software proposta por \citeonline{Chulani2003}. Vide os trabalhos 
de \citeonline{Palza2003}, \citeonline{Ruiz2005}, \citeonline{Castellanos2005},
\citeonline{Becker2006}, \citeonline{Folleco2007}, \citeonline{Silveira2010}. 
Tendo os trabalhos, enunciados anteriormente, como norteadores, adotou-se como 
hipótese de pesquisa deste trabalho que a \textbf{visualização e a extração 
de métricas de código-fonte em ambientes de DWing suprem os problemas 
encontrados nas ferramentas de análise estática de código-fonte}. 


%-----------------------------------------------------------------------------%





%-----------------------------------------------------------------------------%

\section{Objetivos}

Esta seção apresenta os objetivos gerais e específicos deste Trabalho de 
Conclusão de Curso.

\subsection{Objetivos Gerais}
Sob o prisma da hipótese de pesquisa, há como objetivo geral a proposição e 
construção  de um ambiente de \textit{Dwing}, para extrair e visualizar 
métricas de código-fonte.


%------------------------------------------------------------------------------%

\subsection{Objetivos Específicos}

Os objetivos específicos deste trabalho são:

\newlist{objectives}{enumerate}{3}
\setlist[objectives]{ label* = OE\arabic* -}

\begin{objectives}
	\item Construir o ambiente de Dwing para extração e visualização de 
	métricas de código-fonte utilizando ferramentas de software livre. 

	\item Analisar a qualidade de um software livre em uma estrutura de estudo
	de caso.
  
	\item Incorporar indicadores qualitativos para cada métrica de código-fonte.
	
	\item Facilitar o entendimento das métricas de código-fonte, por meio de 
	apresentação, em formas visuais, das análises obtidas por meio das métricas de código-fonte
	
    \end{objectives}
	


%------------------------------------------------------------------------------%

\section {Metodologia de Pesquisa}

\section{Organização do Trabalho}

Para a primeira fase deste Trabalho de Conclusão de Curso, além desta 
introdução, o texto foi organizado em capítulos. O Capítulo 2 apresenta as 
métricas de software bem como o processo de medição, descrito pela ISO 15939, 
as métricas de software bem como as métricas de código-fonte. O Capítulo 3 
apresenta conceitualmente o ambiente de \textit{data warehousing} (DWing). O 
Capítulo 4 apresenta a proposta de ambiente de DWing construída neste 
trabalho. Por fim, o capítulo 5 apresenta as considerações finais com uso do 
ambiente de DWing na extração e visualização de métricas de código-fonte. 
