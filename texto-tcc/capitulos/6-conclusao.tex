\chapter{Conclusão}


Ao longo deste trabalho de conclusão de conclusão, foram investigados mecanismos para avaliar as métricas de código-fonte, facilitar sua interpretação, aumentar a visibilidade destas com intuito de apoiar decisões técnicas, como por exemplo, a refatoração de código. A fim de se alcançar este objetivo, foi hipotetizado que a utilização de um ambiente de \textit{Data Warehousing} iria trazer a automação ao processo de medição de medição de métricas de código-fonte, com a possibilidade de criar um campo de conhecimento semântico para facilitar a interpretação de métricas de código-fonte e trazer visibilidade necessária às métricas de código-fonte.


A fim de se validar a hipótese, avaliou-se em forma de estudo de caso, o Sistema Integrado de Conhecimento e Gestão (SICG) do Instituto do Patrimônio Artístico Nacional (IPHAN) ao longo do seu ciclo de desenvolvimento no ambiente de \textit{Data Warehousing}. Com a execução do estudo de caso, foi possível observar que a evolução do conjunto de métricas de código-fonte neste projeto, cumprindo assim o objetivo específico \textbf{OE1}.

O ambiente de \textit{Data Warehousing} permitiu a automação de todo o processo de medição das métricas de código-fonte, pois foram automatizados a coleta, transformação dos dados e avaliação das métricas quanto aos intervalos qualitativos, cumprindo assim o objetivo específico \textbf{OE2}.

Foi possível observar, que ao se construir as configurações~"OpenJDK Metrics"~e~"Tomcat Metrics" que a interpretação das métricas de código-fonte pode ser flexibilixada com adoção de padrões convinientes ao software avaliando, cumprindo assim com o objetivo específico \textbf{OE3}.

Quanto a avaliação de indicadores de código-limpo e acompanhamento da taxa de oportunidades de melhoria de código-fonte, observa-se que se contribuiu com dois indicadores de saúde de bom código-fonte. Isto é, o acompanhamento detes pode permitir a equipe de desenvolvimento ou até a mesmo a gestores de projeto, tomar decisões técnicas mais eficientes quanto a decisões que impactam diretamente no código-fonte. Estes dois indicadores não só cumprem com objetivos específicos \textbf{OE4} e \textbf{OE5}, mas contribuem também para se alcançar o \textbf{objetivo geral} do trabalho, pois estes são indicadores que podem apoiar uma decisão técnica de refatoração.

Quanto ao objetivo \textbf{OE6}, foram contabilizadas, ao longo do trabalho, as consutas OLAP necessárias para responder aos requisitos de negócio para cada um dos processos que foram avaliados conforme a Tabela \ref{tab:resultados-OLAP}.

 
\begin{table}[H]
\begin{center}
\input{tabelas/resultados-olap.ltx}
\caption{Total de Consultas OLAP realizadas}
\label{tab:resultados-OLAP}
\end{center}
\end{table}
\FloatBarrier


Ao se observar os dados obtidos na Tabela \ref{tab:resultados-OLAP} e a própria execução do estudo de caso da Seção \ref{sec:execution-case-study}, concluiu-se que o ambiente de \textit{Data Warehousing} conseguiu responder às necessidades estabelecidas por cada um dos requisitos de negócio dos procesos de negócio. 

Por meio da utilização do plugin Saiku no ambiente de \textit{Data Warehousing}, é possível cada uma dessas consultas seja exibida em 8 gráficos diferentes ou em Tabelas. Cada uma das consultas pode ainda conter várias outras consultas OLAP que também responderiam a outras necessidades, contudo este trabalho se ateve a contabilizar apenas as que respondiam aos requisitos de negócio.

Por fim, acredita-se que ao satisfazer tanto a cada um dos objetivos específicos, quanto a objetivo geral do trabalho, contribuiu-se na avaliação da utilização de ambientes de \textit{Data Warehousing} em processos de medição de métricas de código-fonte. 

\section{Limitações}

Como limitações do trabalho, cabe destacar que o ambiente de \textit{Data Warehousing}, especificado neste trabalho, é dependente da ferramenta de análise estática de código-fonte. Isto é, como o Analizo, que foi a ferramenta escolhida, limita-se as linguagens de programação C, C++ e Java.

Como apresentado anteriormente, na Seção \ref{sec:validade-estudo}, a utilização de um estudo de caso por si só não é suficiente para generalizar os resultados dele obtidos \cite{yin2011applications}. Por esse motivo, a utilização do estudo de caso é uma limitação do trabalho.

Quanto aos metadados do Projeto de \textit{Data Warehouse}, apresentados \ref{sec:project-dw}, verfica-se que para mapear corretamente um determinado cenário de código-fonte com um intervalo de uma métrica, escolheu-se colocar duas chaves estrangeiras da Tabela~"Meta\_Metric\_Ranges"~na Tabela~"Meta\_Metric\_Ranges\_Meta\_Scenario". Este fato implica uma limitação do modelo em que não é possível associar mais dois intervalos de  duas métricas de código-fonte a um cenário de limpeza de código-fonte.  


\section{Trabalhos Futuros}

Quanto aos trabalhos futuros, foi possível identificar duas possibilidades imediatas. A primeira possibilidade é uma melhor avaliação de como os padrões de projeto propostos inicialmente na Tabela \ref{tab:cenarios} se correlacionam com as avaliações das métricas de código-fonte e a detecção dos cenários de limpeza de código-fonte. A segunda possibilidade é investigação, com mecanismos estatísticos, com apoio de estudo de múltiplos casos, como que a Taxa de Aproveitamento de Oportunidades de Melhoria de código-fonte pode se compartar a depender como se utilização as boas práticas de código-fonte, como as citadas na Tabela \ref{tab:conceitos}.