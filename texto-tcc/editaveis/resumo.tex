\begin{resumo}

	A qualidade interna do produto de software está diretamente ligada as métricas de código-fonte. Estas por sua vez, quando visualizadas apenas em planilhas decorrentes dos resultados de análise estática de código-fonte não apresentam o grau de visibilidade e interpretação necessários para à tomada de decisão técnica em um determinado projeto de software. Neste trabalho, buscou-se, por meio de um ambiente de \textit{Data Warehousing}, facilitar a interpretação e avaliar as métricas de código-fonte e os cenários de limpeza de código a fim de se obter mais facilmente dados que pudessem apoiar à tomada de decisão técnica, como por exemplo, a refatoração de uma determinada classe do Projeto. Para se validar o ambiente, foi avaliado em um estudo de caso, o Sistema Integrado de Gestão e Conhecimento (SIGC) do Instituto do Patrimônio Artistíco Nacional (IPHAN). Esta avaliação resultou em 12 intervalos qualitatitivos para 12 métricas de código-fonte, em 2 configurações de métricas que utilizaram softwares de referência. Outro resultado foi a detecção de 317 cenários de limpeza de código-fonte em 914 classes na última \textit{release} do projeto avaliado.

 \vspace{\onelineskip}
    
 \noindent
 \textbf{Palavras-chaves}: Métricas de Código-Fonte. \textit{Data Warehousing}. \textit{Data Warehouse}
\end{resumo}
