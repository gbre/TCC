\chapter{Introdução}


\section {Contexto}
A qualidade do software depende da qualidade do código-fonte, pois um bom código-fonte é um bom indicador de qualidade interna do produto de software \cite{beck2003test} \cite{ISO25023}. Portanto, decisões errôneas de desenvolvedores podem gerar trechos de código não coesos, que venham a se desfazer com o tempo e que aumentam exponencialmente a chance de manutenções corretivas onerosas \cite{beck2007implementation} \cite{beck1999}.


Entre as formas de analisar a qualidade do código-fonte está a análise estática de código-fonte, que é uma análise automatizada das estruturas internas do código, que permite obter métricas de código-fonte \cite{Emanuelsson2008} \cite{Wichmann95}  \cite{Nielson:1999} \cite{Sommerville10}. Alguns trabalhos como \citeonline{marinescu2004quantifying}, \citeonline{marinescu2005measurement}, \citeonline{moha2006automatic}, \citeonline{moha2008domain}, \citeonline{moha2010decor} e \citeonline{rao2007detecting} mostraram que é possível a partir da análise de métricas de código-fonte, obter indicadores de pedaços não coesos e com alto acoplamento. Estes pedaçoes são passíveis de eliminação com a refatoração, que é a técnica de modificação das estruturas internas do código-fonte sem modificação o comportamento externo observável \cite{fowler1999refactoring}.


\section{Problema}


Embora a refatoração seja uma prática bastante documentada e utilizada principalmente nos processos de desenvolvimento que utilizam métodos ágeis \cite{beck1999}, a decisão de aplicação, quer seja em uma classe, módulo ou projeto, envolve a avaliação de fatores como custo, prazo, risco e qualidade do código-fonte \cite{yamashita2013assessing}.  

Este último fator, que pode ser determinante para decisão de refatorar, normalmente, é díficil de ser avaliado. Isso ocorre, pois as ferramentas de análise estática de código-fonte, normamelmente, não apresentam associação entre os resultados númericos e a forma de intrepretá-los, isto é, as métricas frequentemente mostram apenas valores númericos isolados \cite{Meirelles2013}. Além disso, os resultados obtidos a partir de ferramentas de análise estática de código-fonte são apresentados em longos arquivos (planilhas, arquivos JSON, XML), ou seja, sem mecanismos de separação, agregação e formas de visualização de dados que permitam a fácil identificação de um ponto de melhoria no código-fonte.


%-----------------------------------------------------------------------------%

\section{Questão de Pesquisa}

Tendo em vista que este problema afeta a avaliação de projetos de software, pois é crucial que informações sobre o desenvolvimento de software sejam coletados e compartilhados entre projetos e pessoas em uma visão organizacional unificada, para que determinada organização possa compreender o processo de medição e monitoramento de projetos de software e, consequentemente, se tornar mais hábil e eficiente em realizar atividades técnicas relacionadas ao processo de desenvolvimento de software \cite{Chulani2003}, foi elaborada a seguinte questão de pesquisa:

\textit{\textbf{Como aumentar a
visibilidade e facilitar intepretação das 
métricas de código-fonte
a fim de apoiar a decisão de refatoração
de uma equipe de desenvolvimento?}}


\section{Objetivos}

Colocando a resolução da questão de pesquisa como objetivo geral do trabalho, foram elaborados objetivos específicos utilizando o GQM \cite{Basili96b}. Nesta abordagem cada objetivo específico foi derivado em uma série de questões específicas que são respondidas com métricas específicas, tal como se mostra na Tabela \ref{tbl:obj}. 

\begin{table}
\centering
\input{tabelas/objetivos.ltx}
\caption{Objetivos Específicos do Trabalho}
\label{tbl:obj} 
\end{table}

\section{Hipótese}

Visando atingir os objetivos específicos estabelecidos na Tabela \ref{tbl:obj}, realizou-se uma revisão bibliográfica da literatura em busca de ambientes de informação que visassem à automação de processos e exposição de informações em âmbito gerencial.
Alguns trabalhos como \citeonline{Palza2003}, \citeonline{Ruiz2005}, \citeonline{Castellanos2005}, \citeonline{Becker2006}, \citeonline{Folleco2007}, \citeonline{Silveira2010}, mostraram que ambientes de \textit{Data Warehousing}, são boas soluções para adoção de um programa de métricas em processos de desenvolvimento de software.

Em conformidade com o referencial teórico enunciado anteriormente, foi hipotetizado a hipótese H1:

\textbf{H1: A construção de um ambiente de \textit{Data Warehousing} traz automação necessário ao processo de medição de métricas de código-fonte, possibilidade de criar um campo de conhecimento semântico para facilitar a interpretação de métricas de código-fonte, trazer visibilidade necessária às métricas de código-fonte e inferir outras informações advindas destas.} 

Tendo em vista a adoção da hipotése, adicionou-se o objetivo OE6:

\begin{table}
\centering
\input{tabelas/objetivo6.ltx}
\caption{Objetivos Específico OE6}
\label{tbl:obj} 
\end{table}


%------------------------------------------------------------------------------%

\section{Organização do Trabalho}

Após a leitura da introdução, encontrar-se-á adiante o Capítulo 2 que apresenta as 
métricas de software, processo de medição da ISO 15939, 
as métricas de código-fonte, código-limpo e os cenários de limpeza de código-fonte; O Capítulo 3 que apresenta conceitualmente o ambiente de \textit{Data Warehousing}. O Capítulo 4 que apresenta a projeto  a implementação do ambiente de \textit{Data Warehousing} para métricas de código-fonte e cenérios de código limpo construído neste trabalho. O capítulo 5 apresenta o projeto de estudo de caso do uso como forma de validar a utilização do ambiente. Por fim, o capítulo 6 descreve os resultados obtidos com aplicação do ambiente em forma de estudo de caso, onde foi acompanhado as 24 releases do software SICG (Sistema Integrado de Controle e Gestão) do Instituto de Patrimônio Histórico e Artístico Nacional (IPHAN). Adicionalmente no capítulo 6, são discutidos as conclusões, limitações do trabalho e trabalhos futuros.   