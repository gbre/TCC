\chapter{Solução}
\section{Arquitetura da Solução}
\section{SonarQube}
O SonarQube~\footnote{Toda a documentação e \textit{downloads} estão disponíveis
em \url{http://www.sonarqube.org/}}, anteriormente conhecido como Sonar, é uma 
ferramenta que realiza coleta de dados do código-fonte. Em sua documentação, 
são citados sete eixos de controle de qualidade cobertos na ferramenta: 

\newlist{qualityindex}{enumerate}{4}
\setlist[qualityindex]{ label* = \arabic* -}

\begin{qualityindex}
\item Duplicação de Código
\item Detecção de Complexidade de Código
\item Padrão de Codificação
\item Testes Unitários
\item Arquitetura e Projeto
\item Potenciais Defeitos
\item Comentários
\end{qualityindex}

%--------------------------------------------%
\subsection{Escolha do SonarQube}
O SonarQube foi escolhido como ferramenta para extração de dados do código-fonte
 devido as seguintes características:
\begin{description}
  
\item [Ferramenta Livre:] O SonarQube é uma ferramenta livre que está 
licenciada sob a GNU LPGL 3, que é uma das licenças de software livre.

\item [Suporte a várias Linguagens de Programação:]  É realizada coleta em 
mais de 20 linguagens de programação, como por exemplo, Java, C\#, Python, 
C++, .Net, PHP e entre outras.

\item [Multiplataforma:] Há suporte para os sistemas operacionais Windows, 
Linux, Mac OS e há também para servidores de aplicação Java, como por exemplo,
o Apache Tomcat. 

\item [Integração com Ferramentas de Integração Contínua:] é possível realizar
a integração com ferramentas de integração contínua, como por exemplo, 
Jenkins. Assim a cada versão bem sucedida de construção do código-fonte, é 
possível aferir a qualidade do código.

\item [Persistência em Base de Dados:] toda a coleta de dados é persistida em 
uma base de dados. Há suporte para as principais ferramentas de bancos de 
dados utilizadas no mercado: MySQL, PostgreSQL e Oracle.

\item [Interface Amigável:] Diferentemente de outras ferramentas disponíveis 
no mercado, o SonarQube tem interface web, ou seja, é possível visualização 
das métricas do código-fonte seja realizada em navegador a escolha do usuário.
É possível ainda visualizar todo o código-fonte analisado.

\item [Integração com Ferramentas de Gestão de Defeitos:] É possível integrar 
com ferramentas de gestão de defeitos, tais como Mantis e JIRA.

\item [Suporte a Muti produtos:] Há suporte para visualização de métricas de 
 mais um Produto.

\item [Suporte a Multi-Idiomas:] Há suporte para Inglês, Japonês, Russo, 
Francês, Italiano, Espanhol e Português.

\item [Diversidade de Plugins:] O SonarQube permite a 
extensão de funcionalidades por meio de plugins, sendo que há uma variedade de 
plugins para mais diversas necessidades disponíveis para \textit{download}. 
Cabe ressaltar que alguns são gratuitos e outros tem licença comercial.

\item [Presença de Web Service:] Há um Web Service que extrai as métricas 
disponíveis para diversos formatos, tais como JSON, XML e CSV.

\end{description}


%--------------------------------------------%


\subsection {Limitações Observadas}
Após uma análise ostensiva na ferramenta, concluiu-se que o SonarQube apresenta 
as seguintes limitações:

\begin{description}

\item [Ausência de Visão Consolidada de Métricas:]  A maior parte das métricas
que SonarQube estão apenas no âmbito de Classes ou Pacotes.

\item [Impossibilidade Definição de Múltiplos Intervalos:]  O sonarQube não 
permite a configuração de múltiplos intervalos. É definido apenas, um 
intervalo de alerta (amarelo) e um intervalo de atenção (vermelho).

\item [Ausência de Avaliações Qualitativas:] As métricas não trazem nenhuma 
avaliação qualitativa associada, ou seja, sem o conhecimento específico do 
significado de cada métrica, a interpretação dos valores obtidos não é 
representativo nem elucidativo sobre a qualidade do código.

\item [Métricas como Violações:] O SonarQube utiliza para coletar dados de 
código-fonte, em algumas linguagens de programação, coletores que são 
identificadores de padrões de codificação. Isso quer dizer que algumas métricas 
não são apresentadas com valor absoluto, apenas são indicadas como violações à 
regras estabelecidas pelos padrões. 

\end{description}




\section{Pentaho}

O Pentaho é um software de código aberto para inteligência empresarial, 
desenvolvido em Java pela Pentaho Corporation, que apresenta soluções cobrindo 
as áreas de ETL, \textit{reporting}, OLAP e mineração de dados 
(\textit{data-mining}). 


 
